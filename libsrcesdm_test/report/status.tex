According to the classification presented in Tables \ref{tab:nc_test4_1}, \ref{tab:nc_test4_2} and \ref{tab:nc_test4_3}, the 78 {\bf Working} tests were used to demonstrate ESDM functionalities. There are now five categories in which each test was classified according to their status:

\begin{description}

\item[Success] It means the original test is successful when using ESDM.

\item[Partial Success] It means the original test is successful when using ESDM, but some parts of the test code had to be removed because ESDM does not support that specific feature.

\item[Inconclusive] It means the original test is not successful when using ESDM, and some features still need to be implemented or revised to provide the expected result. The missing features should be available in the short term.

\item[Failure] It means the original test is not successful when using ESDM. Some of the missing features should be available in a medium-term and others are not expected to work with ESDM at all.

\item[Not Tested Yet] It means the original test was not tested yet. This category usually represents tests that are too long to run, tests that demand a large amount of memory or processing or tests involving features that are to be implemented.

\end{description}

Tables \ref{tab:nc_test4_status_1} and \ref{tab:nc_test4_status_2} presents the current status of each test. Table \ref{tab:nc_test4_status_3} summarises the results.
