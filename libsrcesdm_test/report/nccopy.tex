\chapter{Tests with nccopy}
\label{ch:nccopy}

\section{Introduction}

{\itshape
The nccopy command-line utility copies and optionally compresses and chunks netCDF data.

The nccopy has options to specify what kind of output to generate and optionally what level of compression to use and how to chunk the output.
}\footnote{Reference: \url{https://www.unidata.ucar.edu/software/netcdf/workshops/2011/utilities/Nccopy.html}}

The nccopy utility can be found in the directory

\begin{framed}
esdm-netcdf/build/ncdump/nccopy
\end{framed}

and the simplest way to call is

\begin{framed}
nccopy input\_file output\_file
\end{framed}

As we want to test if the files generated by ESDM are compatible with NetCDF, we are going to run four different tests, described in the next sections.
To call ESDM, just insert \texttt{esdm:$\backslash\backslash$} before the file.

Before starting the tests, we need to copy or link the file \texttt{\_esdm.conf} to the directory \texttt{esdm-netcdf/build/ncdump/nccopy}.
This can be done using a copy from the directory \texttt{esdm-netcdf/dev}.
It is also good to run the \texttt{mkfs.esdm} utility prior to the tests with the following parameters.

\begin{framed}
\$ mkfs.esdm {-}{-}create {-}{-}remove {-}{-}ignore-errors -g -c \_esdm.conf
\end{framed}

\section{Files}

Table \ref{tab:netcdf} introduces information about two originally NetCDF files\footnote{Reference: \url{https://www.unidata.ucar.edu/software/netcdf/examples/files.html}} that will be used for testing.
For simplicity, these files are renamed using their sizes as reference (column Nickname).

\begin{table}[H]
\centering
\begin{tabular}{|l|l|l|m{4.2cm}|}
\hline
\multicolumn{1}{|c|}{NetCDF File}	& \multicolumn{1}{c|}{Nickname} & \multicolumn{1}{c|}{Size} & \multicolumn{1}{c|}{Description} \\ \hline \hline
\texttt{sresa1b\_ncar\_ccsm3-example.nc} & \texttt{small.nc} & 2.8 MB & From the Community Climate System Model (CCSM), one time step of precipitation flux, air temperature, and eastward wind. \\ \hline
\texttt{test\_echam\_spectral.nc} & \texttt{big.nc} & 281.4 MB & Example model output from the ECHAM general circulation model. Almost CF, but not quite. Has a spectral coordinate for variables such as temperature (st) and vorticity (svo). \\ \hline
\hline
\end{tabular}
\caption{\label{tab:netcdf} Sample files following CF conventions}
\end{table}

\section{Tests}

Copy the original NetCDF file to ESDM:

\begin{framed}
nccopy small.nc esdm:$\backslash\backslash$small-out.nc
\end{framed}

Copy the ESDM file to NetCDF format:

\begin{framed}
nccopy esdm:$\backslash\backslash$small-out.nc small-final.nc
\end{framed}

Compare the initial and final files:

\begin{framed}
diff small.nc small-final.nc
\end{framed}

The absense of output using the command \texttt{diff} comproves the files are the same.
The procedures are the same considering the file \texttt{big.nc} which is also perfectly copied.

We can also check the metadata of the ESDM file, which is available inside the directory \texttt{\_metadummy/}.

\begin{comment}

PLEASE, CHECK.

The good news is that now the files remain with the same size. In previous versions, we were having different metadata, which was leading to different size in the final file.

The information that was previously in this report cannot be reproduced anymore. The metadata is huge, even for the small.nc file. The name of the file also changed now, and it is starting with \, which is weird.

For the nccopy tool, we still have some undesired outputs.

==>> nccopy small.nc esdm:\\small-out.nc

The dimensions:

0 = 128
1 = 256
.
.
.
.

and the final message from ESDM, that eventually have to be fixed.

ESDM has not been shutdown correctly. Stacktrace:

Those extra messages I don't understand, but I think they are related to the system.

3: nccopy(+0x8c18) [0x557e0f680c18]
4: /lib/x86_64-linux-gnu/libc.so.6(__libc_start_main+0xe7) [0x7fc642442b97]
5: nccopy(+0x29ea) [0x557e0f67a9ea]

==>> nccopy esdm:$\backslash\backslash$small-out.nc small-final.nc

Error in ESDM_inq_var_all, several times:

[ESDM NC] WARN ESDM_inq_var_all():1928. ESDM does not support compression!

and the same final message from the previous case.

- - -

The original idea was to have something like the following code in the report, but with these files this was not possible.

lucy@lucy:~/esiwace/esdm-netcdf/build/ncdump/_metadummy/containers$ more smallfile_out.esdm.md
{"Variables":{"type":"o","data":null,"childs":{"_nc_dims":{"type":"q1@m","data":["lat"]},"_nc_sizes":{"type":"q1@i","da
ta":[1]}}},"dsets":[{"name":"lat","id":"eUyIKbbzj9Y0kEpk"},
{"name":"lon","id":"eUyIKbxOPzPXyKqL"},
{"name":"Elevation","id":"eUyIKbrJHZSAAnbt"}]}

lucy@lucy:~/esiwace/esdm-netcdf/build/ncdump/_metadummy/containers$ python -m json.tool smallfile_out.esdm.md
{
    "Variables": {
        "childs": {
            "_nc_dims": {
                "data": [
                    "lat"
                ],
                "type": "q1@m"
            },
            "_nc_sizes": {
                "data": [
                    1
                ],
                "type": "q1@i"
            }
        },
        "data": null,
        "type": "o"
    },
    "dsets": [
        {
            "id": "eUyIKbbzj9Y0kEpk",
            "name": "lat"
        },
        {
            "id": "eUyIKbxOPzPXyKqL",
            "name": "lon"
        },
        {
            "id": "eUyIKbrJHZSAAnbt",
            "name": "Elevation"
        }
    ]
}
\end{verbatim}

\end{comment}
